
\chapter{Adventuring}

\section{Environmental hazards}

\label{hazards}

\rpgart{t}{img/art/galleon}

The environment can and will have an impact on your actions. In this section, we detail some considerations, both from a gameplay and a plot perspective.

\subsection{Terrain}

When it come to environmental hazards, here is the general rule: when entering a hazardous terrain, the world rolls to apply a Condition against you, much like a trap would. The precise nature of the roll is left to the GM, but will usually involve something like Athletics against a predefined Difficulty. Some penalties (or positive effects) are unavoidable: nobody can see colors in absolute darkness, and the density of water will always slow projectiles compared to air. 

On average, a difficult terrain should apply a -2 modifier to most Actions, but that is for the GM to judge. It can also reduce Defenses: it is easier to hit someone who is mired knee deep in mud. Conversely, some Actors may also be used to those conditions and immune to a given penalty: an adept swimmer will not be as penalized underwater.

Remember that terrain, and the world in general, can be interacted with. It is very possible ot use it to make elemental Combinations (see section \ref{combinations}). Fire melts ice, water conducts electricity, oil is flammable, etc. Fling stuff at your opponent, and see what happens! This includes out-of-combat interactions.

Mobility Abilities (ie. Into The Fray) still work if you could physically make the move without using an Ability.


\subsection{Dealing with Nature}

When considering the penalties and limitations to give to your (presumably human) Player Characters, here are some things to keep in mind.

A regular human can jump about 2 meters far, and 50 cm high (a bit more by running to gain momentum), although records are much higher than that. MIG and Atheltics are relevant here ; they are also relevant to determine what happens when moving heavy objects
\footnote{The action-reaction principle dictates an opposing response to an applied force. Total momentum (mass times speed) is conserved in elastic collisions; in reality however, there will be some small loss of momentum.}
.

Fluids such as water (and air to a lesser extent) have friction. Heavy Armor will sink you, and most people can only hold their breath for a minute or two (depending on CON). Diving deeper than a dozen meters, or breathing freely at altitudes above 3km will require magic. And at more extreme heights or depths, pressure will become a problem.

Temperatures outside the 0°C-40°C range will require CON tests, and when pushing further ambient temperatures above 90°C or below -30°C can only be survived for minutes without equipment. All elements discussed later, such as how much food is needed per day, still apply.

Many biological substances are poisonous to humans\footnote{Putting small doses on scratched skin to test for an allergic reaction can be a useful litmus test. But many plants, fruits, grubs, animals and rainwater are edible if boiled properly.}. Poisons, and illnesses in general, will be countered by Fortitude: this will require repeating a Fortitude roll (with the Difficulty depending on the illness) at regular intervals, with the illness progressing to a worse stage if the roll is failed.

Fall damage is 5 HP per 3 meters fallen\footnote{Gravity is a force, and obeys the laws of Newton on the non-quantum scale: $F=m\bm{a}$ and $\bm{a} = \frac{d\bm{v}}{dt}=\frac{d^2 \bm{x}}{dt^2}$. In reality, atmospheric friction will limit fall speed to the terminal velocity, which is not considered here for simplicity. Since acceleration is a change in velocity, those rules also explain ballistics when initial speed in taken into account.}. An open flame will give 5 damage per round\footnote{Fire, meaning combustion, is a rapid oxidation. It is also an example of an exothermic reaction.}, the same for suffocation.


\section{Resting} 

A short rest restores all Hit Points up to the next Wound (see section \ref{wounds}) threshold. Recall that the Wound thresholds are 25\% and 50\%: this means that if you are at 75\% of your maximum HP you will fully heal in a short rest, but if you are at 10\% you will heal up to 25\% only. A long rest, medicine, and/or healing spells are required to cure Wounds. 

In terms of duration: \textbf{short rests} occur between scenes, when you can spend a little time to catch your breath. This is usually a few minutes to a few hours, depending on the GM. On the other hand, \textbf{long rests} occur between chapters of a story act, when you have some downtine to recuperate, craft, etc. Usually one (or a few) nights, but it can be more or less than that as long as it marks a signficant downtime.


\subsection{Morale Wounds}

Hit Points do not represent only an Actor's physical well-being, their mental state plays a role as well. For instance, recall that upon taking your second Wound, you need to succeed in a RES test to continue fighting. This also means that the Wounds ganed when losing HP can be not only physical, but mental in some cases: panic, trauma, perhaps even PTSD, gaining manias, etc. 

Suffering or inflicting particulary gruesome acts, as well as feelings of helplessness, should trigger this type of Wound instead of a physical one. But these should remain an exception, a setpiece moment, and be used sparingly by the GM. I recommend allowing a RES test to avoid the more severe ones, and correspondingly failing an important RES test may inflict such a Wound.

Some people, as Traits or consequences of past actions, may have "mental resistance" to certain sources of stress. Seeking help from healers or loved ones (thus straining the relation) can help in healing from mental Wounds.


\subsection{Grievous Wounds}

When gaining your second Wound, depending on the circumstances, the inflicted Wound has a small chance to be a Grievous Wound instead. Such wounds inflict a relevant Condition until cured, instead of a flat -1 to every roll. For example, a wrenched articulation will reduce your mobility, a concussion will reduce you mental faculties, frostbite will hamper your dexterity, etc.

When Knocked Out, you automatically gain a Grievous Wound on top of the two Wounds you already suffer.

While light cuts will close in a few days, such Grievous Wounds can take weeks or months to heal back to function, and may require re-education over a year. Medical supplies will be needed daily. Relatedly, a blood loss of 1-2 liters is survivable, but more is dangerous. Aging will diminish somewhat your physical abilities, and your ability to recover from such injuries.

\section{Experience}

\label{experience}

Experience (XP) is not granted by defeating enemies. Instead, the GM grants it to the players to represent a character's achievements, or reaching story milestones. I recommend awarding the same XP at the same time for all players.

The gained XP can be spent during any long rest. The only exception is if spending it to regain a Heroism point, which happens instantly and can be done at any time, but no more than thrice between long rests. Unspent XP is kept and can carry over\footnote{Indeed, I would encourage you to spend 50-60\% of your XP when leveling up and save the rest for Characteristic increases}. XP costs and the associated boons are presented below:

\begin{rpg-table2}[Xc]
	\textbf{Boon}  & \textbf{XP cost} \\
	Increase a Skill by 1 (or get the first point) & 35 \\
	Increase a Special Skill by 1 & 25 \\
	Increase a Skill or Special Skill by 1, if it was already $\geq 6$ & 50 \\
	Learn one new Ability & 50 \\
	Increase a Characteristic by 1 point & 50 \\
	Increase a Characteristic by 1 point if it was already $\geq 20$ & 100 \\
	Increase maximum Focus by 1 & 25 \\
	Regain a Heroism point & 5 \\
\end{rpg-table2}

Those costs are a rough estimate and can be tweaked up or down if necessary. See section \ref{leveling_balancing} for what to aim for when deciding which boons to allow or disallow when leveling up, as well as how often to grant XP.

Note that your choices in increasing Skills (and Abilities) must be meaningful and based on the skills you actually used, or learned. You may gain new Special Skills when leveling up by discussing it with the GM, but again it must be rooted in the plot. When deciding what to upgrade when leveling, poetically speaking, I would recommend that you think about optimizing for "survival of the fittest": the goal is to reinforce elements that contribute to your objective, be it roleplay or efficiency.

Furthermore, for every 100XP you spend in total, you \textbf{gain a level}. A player character may not spend more than 1000 XP total, corresponding to a level of 10. This is not purely cosmetic: at any time, a player's maximum HP is equal to the PC's $Constitution \times (1.75 + 0.25 \times L)$, rounded \textbf{up}. Note that you use the raw numerical value of Constitution, not the modifier.

A few additional restictions apply\footnote{Of course any restriction can be waived by the GM if they so desire.}. Let $L$ be a PC's level (a PC begins at level 1, not level 0) in the following:

\begin{itemize}
	\item Maximum Focus cannot be higher than $5+L$
	\item Skill modifiers cannot be higher than $5+L$
	\item No Characteristic can have a base value above 20 until $L \geq 5$
	\item A Characteristic can not be increased by more than 1 per level up\footnote{Formally: a Characteristic base value is capped at $S + L - 1$ where $S$ is its value at character creation.}.
	\item A character cannot spend more than 9 Ability Points (to learn new Abilities) lifetime, including the starting 4 ; this does not include Mythic Abilities.
	\item Abilities must be learned in order of their tree\footnote{As mentioned before, the GM may let a PC break this rule, especially at character creation, but this should remain an exception.}. Certain Abilities may need to be taught however, at the GM discretion.
\end{itemize}


Finally, at level 5, Player Characters unlock a Mythic Ability (see section \ref{signature}). A second is unlocked at level 8. Note that level 1 characters are not necessarily "ordinary", they can have some backstory and be already quite potent. Relatedly, although I recommend beginning at level 1 in most case, campaigns do not need to end at level 10. Only spanning epics should: a level of, say, 5 is a perfectly acceptable stopping point.

\paragraph{Teaching and Learning}

Someone who knows more than you do on a topic can teach you, or you can gain new knowledge yourself if you have access to the approriate books. It may take a few months, though. You may also improve through scientific experimentation, and more generally trial and error, but creating new knowledge is slower than being taught it...


\section{Equipment}

\label{equipment}

A human Actor can carry 5 kg per Might point. Mules or horses can carry 200+ kg, and wagons even more.

It is assumed that each PC possesses basic equipment and supplies. This includes blankets, cooking pots, a few short ropes, short knives, a sewing kit, eating utensils, and other basic tools. Other equipment that is not necessarily assumed and must be tracked include rations, long ropes and grappling hooks\footnote{Grappling hooks are versatile objects that can also be used in combat (they are indeed not the only objects that can be used in combat also), with the rope reacting to the weight exerted upon it. An elastic object streched by $x$ will try to return to its equilibirum position by exerting a force of $\frac{1}{2}kx^2$ on its attachment.}, tinderboxes, and thieves' tools. Their price is commonly a few silver coins, depending on the local economy. Recall that torches burn for 1 hour over a distance of 10 m, but lanterns light up 30m cones.

The average person will need to consume a minimum of 1 kg of food and 2 liters of water daily, and can need double that amount when they are physically active or if the conditions are taxing, and might need a bit more water for hygiene. Said person may last a few weeks on half rations, but more than one week without water or one month without food is almost never survivable. Relatedly, one cannot survive more than a few days without sleep.


\paragraph{Tools}

In certain cases, the GM may decide that you need a tool to perform a desired Action. Without it, the Difficulty is increased, while conversely an excellent tool may grant a bonus. 

For example, let's say you want to pick a lock. If you have a lockpick, no penalty is applied. But if you have only a stiff wire, or a dagger, a penalty of -3 is warranted. Conversely, using a crowbar to force a door instead of ramming it yourself gives you +3, trying to find herbal components in a luxurious forest is easier, etc. The possibilities are endless.


\paragraph{Lifestyle} 

For simplicity, in most of the world, let us consider that one gold piece equals 10 silver pieces and 100 copper pieces. Copper pieces are the basic denomination, abbreviated to "cp". As a general rule, most of the world gets less than 100 cp a month, the middle class has from 200 cp to 2000 cp per month, and anyone above is rich. This is of course variable depending on local conditions.

On the character sheet, the wealth of a character is tracked in cp. The wealth held in small change is separated from the wealth held in golden coins and gems. This is because the former can be spent without suspicion but is bulkier, while the latter is more convenient but can attract unwanted attention.

The general price of goods and services will obviously vary enormously by region, but as a general rules most common services will be paid in copper or silver, and most real estate will rent yearly for silver or at worst gold.

\subsection{Armaments}

Armaments include Weapons and Armors. It should be noted that Armaments, and by extension most equipment, have a Tier: higher quality ones will be more effective (more damage, more protection, etc.). This is however usually not relevant until later in the campaign, and as such is explained later in section \ref{balancing_equipment}.

\subsubsection{Weapons}

The damage inflicted by a weapon on a successful Attack depends on its class (also known as its type). The general class breakdown is given in Table \ref{weapon_damage}. 

Light Weapons include daggers and small clubs. Heavy Weapons include greatswords, muskets, and warhammers. Regular (also called Medium) weapons form the remainder, such as swords, bows and light crossbows.

\begin{table}[h!tbp]
	\begin{center}
		\begin{tabular}{p{1.25cm}p{1.5cm}p{4cm}} \toprule
			
			\textbf{Type} & \textbf{Damage} & \textbf{Other} \\ \midrule
			
			Light & 3 & +3 Accuracy \\
			Regular & 5 & \\
			Heavy & 7 & Only usable once per turn, but ignores 2 Armor\\

			\bottomrule
		\end{tabular}
	\end{center}
	\caption{Weapon damage}
	\label{weapon_damage}
\end{table}

\paragraph{Clarifications}

The +3 Accuracy bonus given by Light weapons does not stack if you wield two of them. Furthermore, since Overwatch technically consists of a delayed Action, you cannot Overwatch with a Heavy Weapon if you used it during your turn.

\paragraph{Special Weapons} 

Special Weapons can be granted by the GM, which deviate from these templates. For instance, gunpowder weapons pierce 1 additional Armor for their class: a Heavy musket will pierce 2+1 Armor. See section \ref{balancing_equipment} for more ideas.

\subsubsection{Armor}

\label{armor}

As discussed in the Combat chapter, each Armor point is a flat reduction of one to all incoming Damage. Light (gambeson, ...), Medium (mail, ...) and Heavy (plate, ...) Armor gives respectively 1, 3, and 5 Armor. Light Armors are very widespread, but Medium Armors will usually be worn by seasoned fighters, and Heavy Armor should be restricted to elite foes for game balance reasons. 
Anyone can wear Light or Medium Armor, but wearing Heavy Armor requires training.

Armor may give different protection against different types of Damage. Indeed, in the base rules, elemental and magical damage ignores Armor, and gunpowder is particulary effective against it\footnote{It's possible to have certain types of Armor protect against some types of elemental damage, or protect more against slashing than blunt damage, or anything you want. But, in general, I recommend sticking to the simple rules as currently written.}. Shields do not grant Armor, but give +3 Deflection instead. A particularly large shield may give more Deflection but reduce Accuracy.

However, wearing Armor will inflict some penalties as detailed in Table \ref{armor_effects}. They will increase your Fatigue (see section \ref{resources}), reduce your Discretion and impose a "soft cap" to your Dexterity and Wits: points above the soft cap are halved (rounded down)\footnote{For example, wearing Medium Armor: a Wits of 16 is still 16, but 17 is reduced to an effective value of 16, 18 and 19 is reduced to 17, 20 is redued to 18.}. Do not let those penalties discourage you! In fact, in most cases wearing suitable Armor is absolutely paramount for a Player Characters, and the benefits outweigh the costs.


\begin{table}[h!tbp]
	\begin{center}
		\begin{tabular}{p{1.2cm}p{1.1cm}p{1.2cm}p{1cm}p{1.7cm}} \toprule
			
			\textbf{Type} & \textbf{Armor} & \textbf{Fatigue} & \textbf{Soft cap} & \textbf{Discretion penalty} \\ \midrule
			
			Light & 1 & 0 & N.A. & 0\\
			Medium & 3 & 1 & 16 & -3 \\
			Heavy & 5 & 2 & 14 & -6 \\

			\bottomrule
		\end{tabular}
	\end{center}
	\caption{Armor effects and requirements. The soft cap its to Dexterity and Wits, and will halve the points above it.}
	\label{armor_effects}
\end{table}



It should be noted that Armor can technically be negative, increasing incoming damage; this is mostly relevant for Atrophy and Vulnerabilities.


\subsection{Crafting}

Crafting can be an important part of certain Player Character builds. In particular, Magic users and characters relying on gadgets and contraptions will make good use of it. A large variety of objects are covered by Crafting. This includes, but not limited to, potions and poisons (lethal but also non-lethal like chloroform), prepared Spells inscribed on scrolls, bombs (smoke, explosive, etc.), traps of any sort, and so on so forth.

Throughout the manuscript, crafted objects are sometimes referred to as Gadgets. The GM will probably work in case-by-case on this when deciding if the Gadget proposed by the player is actually doable, or when proposing Gadgets to the player, but here are some general guidelines. 

\label{crafting}

Crafting (and Enchanting, by extension) are time-consuming processes and usually require a long rest. You can craft any simple object if you have the necessary materials or ingredients, and the blueprint memorized or at hand. However, the most powerful upgrades, devices and enchantments should require special, hard-to-acquire ingredients such as monster parts, plants and gems, and possibly special tools. 

If a player wishes to craft a special contraption, they will describe it. The GM will adjudicate if it is possible, the properties (in terms of game mechanics) of the crafted object, and determine the Difficulty of the crafting roll. The PC's Aptitude for the crafting roll depends on what exactly is being crafted, but it will generally be along the lines of INT + Crafting. INT can be replaced by WIT in certain circumstances. Crafting is usually replaced by Kosmics for Enchantments.

To determine a Gadget's gameplay effect, the general principle is the same as Spell crafting, which we will discuss later in section \ref{spells}. As such, I would recommend familiarizing yourself with Spell creation and balancing before tackling Crafting, since their principles are similar. I do however give a brief summary below. 
Some examples of crafted objects are presented in section \ref{examples}! Feel free to use them as inspiration.

This also means that using a Gadget (or any Object) will usually demand an Ability roll with an Accuracy, see section \ref{accuracies}.

In practice, the players should expect Gadgets to be slightly more powerful than regular Abilities, but with more limited uses: indeed, the main balancing factor of Crafting is that you can efficiently \textbf{craft or enchant a number of items per long rest only equal to your Crafting or Kosmics respectively}\footnote{The crafting rolls themselves are not as important a limitation, in terms of balancing, and should often succeed, at least for run-of-the-mill Gadgets.}. 


Additional balancing considerations, for both Crafting and Enchanting are detailed for the GM in section \ref{balancing_crafting}.


\paragraph{Difficulty} 

The Difficulty to create an object is equal to the \textbf{number of Accents required} to make a roughly equivalent Spell, that is to say a Spell that would have the same gameplay effect. See section \ref{spells} for more details, but for now it is sufficient to remember that, roughly, adding 2 to a numerical effect (usually damage or healing) or inflicting a modifier of \pm 1 will each require one Accent. Critical Successes can impart special properties to objects if the GM wishes.

For example, when crafting a trap that would inflict 7 damage\footnote{It's 7, not 4, since the first Power Accent is free and applies 3 damage} or apply a modifier of -3 to a single enemy, and the trap has a range of a few tiles: that is roughly equivalent to a spell with 3 Accents (targeting + 2 Power) so the Difficulty will be 3.

Relatedly, if you make a "trap" (in the general sense, magical or mundane), its Difficulty when encountering it and trying to avoid its effects equals the Aptitude that was used to craft it, as usual for the rules. It can be higher than your crafting Aptitude if you scored a Critical Success. 

For objects that only relevant to the plot (that do not have a direct gameplay effect), the Difficulty is set by the GM based on the odds they want to give (see section \ref{balancing_rolls}), and what the crafter should reasonably be able to do. 

\paragraph{Restocking}

As mentioned, on each long rest, the number of Gadgets you can craft is equal to your Crafting skill\footnote{The GM can increase or reduce this number depending on circumstances, as usual. You know the drill by now.}. Once you have managed to craft an Gadget, you can craft new copies again to replenish your inventory; this is known as "restocking". Restocking is easier than the initial Crafting: indeed, while Restocking still requires rolls (to determine if you produce a normal, critical, or worse version of the Object), a Failure only results in a worse object. You only fail completely on Fumbles.

As such, Restocking is a reliable way to replenish your "bag of tricks", as it were. Of course, this is only applicable for fairly standard Objects and consumables: you cannot Restock a rare or very complex contraption, you must re-roll it every time. Special artifacts and plot items are also excluded for obvious reasons. As always, the GM decides. Most "consumables" should be able to be replenished easily, be it by crafting them yourself or buying them, whenever you get the opportunity to take a long rest. 

To avoid hoarding, do not let players store more than the equivalent of two rests' worth of crafted Objects without serious encumbrance penalties.

\subsubsection{Enchanting}


\rpgart{t}{img/art/machine_head}

\label{enchanting}

An Enchantment is the process of associating a Spell to an object, and is technically a subset of Crafting. A Spell is a combination of a Sigil and Accents, as explained later in section \ref{spells}; I recommend reading this section before going back to Enchanting. Much like Crafting, you can only Enchant as many objects as the value of your Kosmics Skill between each long rest.

There are two distinct cases of Enchanting:
\begin{itemize}
	\item If you are inscribing a Spell on an object that already has existing properties, such as a weapon or a crafted Gadget, you are doing Inscribing.
	\item Otherwise, you are creating a Magical Implement, which is a Gadget that holds a single charge of a Spell.
\end{itemize}
The "Enchanting" broad term refers to both of these, and as such any of the two counts towards the daily Enchanting limit.

When Enchanting, there is still a Difficulty associated with this, fixated by the GM depending on the quality of the Spell used. In fact, Enchanting is technically a sub-type of "Calm Casting" (see sectino \ref{casting_types}). In most cases, the Difficulty is 1 per Accent on the Spell you want to engrave, plus the usual penalty if you are using more accents than your INT. This represents the difficulty of stabilizing Sigils. The quality of the success will affect the quality of the object you make: Criticals in crafting result in more powerful objects, grazes result in ones with drawbacks.
	
Once created, Enchanted objects still require some expertise to use properly. Any Actor attempting to use an Enchanted object should have a Kosmics Skill of at least 1, otherwise all results will be downgraded by one rank (a Success becomes a Graze, a Failure becomes a Fumble, etc.).


\paragraph{Inscribing}

Only objects of sufficient robustness and quality (the GM decides) can be Inscribed, with better quality objects being able to house more powerful Spells. You may Inscribe a Crafted object.

Note that, like Spells, Inscribed Spells are not necessarily permanent. By default, an Inscribed object hold a single charge of its Spell. It is possible to increase the charges\footnote{Refreshing an Enchantment which has completely depleted its Spell charges is not possible, and will require a new Inscribing.} of an Inscribed object by Refreshing the Enchantment, which works exactly like Restocking a Crafted object (easier roll). The number of charges an object can hold depends on its quality, and the maximum number of charges is equal to your Kosmics skill\footnote{For example, someone with a Kosmics skill of 5 can put one more charge on an object which has already 4 charges. Also note that, much like Restocking is still Crafting, Refreshing is still Enchanting and counts towards your daily limit.}

Similarly, the targeting also depends on the Accents of the engraved Spell: for example, a thrown object will need an Impact Accent for range on its Spell.

\paragraph{Magical implements}

Magical implements are Gadgets that only hold a Spell, usually a more powerful one than you would be able to Snap Cast. They hold a single charge, meaning they can be used once. For regular spells, the receptacle can be a simple parchment, slab or wand. The difference between making a Magical Implement and Inscribing an object is that an Inscribed object keeps its properties: an enchanted grenade keeps the proprieties of the original grenade.

When using a Magical Implement, you make a roll as if you were casting the Spell it contains yourself. This usually means making a roll with an Accuracy against a Difficulty. The main difference is that you never take any penalty for exceeding the number of Accents allowed by your INT, since the Spell has been already prepared.


