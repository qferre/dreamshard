
\chapter{Abilities}

In this RPG, there are no pre-defined Character classes. Instead, a Character will learn Abilities by picking them from different Ability trees. This lets each player to create their own custom build.

Abilities generally work by unlocking special, powerful Actions that an Actor can perform. Those are mostly combat Actions, but many can be useful out-of-combat as well. In most cases, \textbf{using an Ability requires spending a Focus point to be used}, as explained below in section \ref{resources}. Some Abilities will instead unlock new features or provide boons to the Actor.

\label{abilities}

Abilties are divided into three colors, each of which is itself divided into three Ability trees. The colors are mostly used as an instinctive sorting, and only have an impact when creating a Player Character: the player should pick a major color and perhaps a minor one as well, and the Mythic Abilities for this character should be designed based on the Abilities from the chosen colors. 

\textit{Red} focuses on controlling the field. \textit{Green} instead favors agility and adaptability, and \textit{Blue} instead seeks to influence the field from a distance.

To pick an Ability, a character must spend XP and already know the previous Ability in the same tree (see section \ref{experience}). If the player make a very convincing roleplaying argument, a few Abilities \textit{might} be granted out of order, but this should be an exception. The first Ability can always be picked.

Abilities associated with an attack will usually work by making a standard Attack and then making a second roll to check whether a supplementary effect is applied. But other Abilities will have their own roll separate to do something completely different. All of this, the conditions and effects of each Ability, is explained in full in the descriptions. The \textbf{detailed list of Abilities} is presented in section \ref{abilities_list}, in the Appendices. 

An Ability always succeeds unless it explicitly calls for a dice roll in its description. Note that hostile Abilities will usually call for one. In contrast, Gadgets and Spells always require a roll unless they specify otherwise, even if they are not hostile, but there are caveats in that case: see section \ref{accuracies} for more details.



\paragraph{Creative uses} 

\label{creative_use}

Abilities can be used in a variety of creative ways, as most of them will have an additional effect beyond simply "dealing more damage". Relatedly, Abilities can be used to effect Combinations, elemental or not, both in-and-out of combat. For the most part, the laws of physics and science as we know them in our own universe also apply here. That said, magic can bend them. Indeed, Spells like the Diviniation Sigil can be extremely useful in an investigation. Elements (created by magical or mundane means) can have varied effects on the world: for example, fire can be used to burn stubborn vines, whether in a fight or simply when exploring.

Indeed, Abilities are not limited to combat. In fact, they can have many out-of-combat uses! This is also true of "combat" Actions in general (including Concentrate), and recall that World Initiative can be used for puzzles, even without combat. For instance, an Ability that allows a friendly Actor to reroll their latest roll can be useful in an important conversation\footnote{For balance, this is why many of those will be limited both by Focus and by a preset number of uses per-rest...}.


\section{Special Abilities}

The GM can grant to any Actor, PC and NPC alike, the possibility to learn Special Abilities outside of the predefined trees, much like Special Skills are tailor-made Skills. However, unlike Special Skills, those should be much rarer. Of course, such Special Abilities can be both combat and non-combat, active or passive, may or may not be Spells, etc. The sky is the limit! But please be sure to see section \ref{balancing_abilities} for advice on how to design and balance them.


\subsection{Mythic Abilities}

\label{signature}

In particular, one staple type of Special Abilities that I recommend using in most campaigns are called Mythic Abilities. Those are very powerful Abilities tailored to the character, designed in collaboration between the GM and the players. To balance them, using them costs \textbf{both an Action Point and a Heroism Point} (see section \ref{heroism}). On the flip side, since their use is limited, the GM can go wild when designing them and make them very impactful. Like Heroism itself, those are designed to be low-frequency but high-impact plays that can turn the outcome of a situation.

When using a Mythic Ability, no roll is required: it always triggers successfully and applies its effects as described in its Ability text. The text of the Ability itself, however, may demand a roll to apply certain effects.


\section{Focus}
\label{resources}

Using an Ability consumes a point of Focus\footnote{The GM may design Abilities that consume more or less: notably, even in the standard rules, Passive Abilities as written now do not cost Focus. Any Ability that has a non-standard cost will say so in its text.}. Focus is regenerated by a short rest. A level 1 character begins with a maximum Focus of 5. This maximum can be increased by spending Experience Points. 

On the other hand, Passive Abilities are always active and do not cost a Focus, unless specified in their text. Most (but not all) active Abilities also cost an Action Point.

\paragraph{Fatigue}

Each point of Fatigue will reduce an Actor's maximum Focus by 1. Recall Armor gives Fatigue. Not having slept in more than 16 hours also gives 1 Fatigue every 3 hours.

Having 3 or more points of Fatigue will also inflict the Weakened Condition until rested.


\section{Magic and Sigils}
\label{spells}

\rpgart{t}{img/art/dark_nun}

Sigils are the core of magic Spells. They are learned by purchasing them as Abilities: namely, they are the Abilities with "Sigil" in their name\footnote{Somebody who is sufficiently educated in magic can attempt Sigils they have not yet mastered, but only as a calm casting (see below) and with hefty penalties. In fact, this should mostly be done for plot reasons.}. However, Sigils by themselves do nothing: to cast a spell, Accents must be added, which will influence the Difficulty of the roll. The combination of a Sigil and Accents forms a Spell, which is added to the Actor's Character Sheet. The Accents influence the Difficulty of the roll made to cast the spell.

The number of Spells in an Actors' repertoire cannot be higher than their 2\texttimes Kosmics. However, not all of those Spells are accessible at once: the number of spells each Actor can keep in memory at a given time equals 1 + Kosmics instead. A long rest is needed for a repertoire change, but changing which spells of the repertoire are presently kept in memory requries only a short rest.

The players should compose their spells beforehand, thinking about what they want to achieve, and the GM will determine if this is acceptable and which Accents would be needed to do so. Otherwise, spells can have any effect as the GM/plot demands.


\subsection{Accents}

An Actor can cast a spell with as many Accents as their INT without penalty\footnote{Negative values still count. An INT of -1 will negate the first free Power Accent, and an even lower INT will apply penalties for each Accent.}. Each \textbf{Accent above this limit gives a -3 malus} to the roll made when casting the spell. This is the main factor ensuring spell balance.

The complete list of Accents is given in section \ref{spell_accents}. The most important ones to remember are the \textbf{Power} accents, which determine the magnitude of the effects, and the targeting Accent(s) which will determine the Spell's target(s). The effects of the Accents, and more generally how to balance Spells, are explained in details in the Appendices; but for now it is sufficient to remember that, roughly, adding 2 damage or inflicting a malus of -1 will each require one Accent, and that the first Power accent is free. 

The description of each Sigil explains the kind of effects that can be expected from it. This can be very broad, and many variant spells can be made with the same Sigil. The GM reserves the right to demande more Power Accents (or even penalty Accents) to obtain a certain effect: for example, telekinesis of a small object and levitation of a person are both covered by the Force sigil, but the latter is more difficult and will require more Accents. Same reaoning for charm, silence and sleep. Also, Fire can be used to burn something, but also to heat it up, to make a fireball, etc. Again, see the Appendices for more details.

The key is to remember that Power Accents balance the magnitude of the modifiers applied, and that a given Spell will usually apply a single Condition of magnitude comparable to the Conditions presented before, or have a numerial effect (damage, healing). Some Spells may do both, but of course with reduced magnitude, and this is usually an exception.

Accents do not exclude one another in most cases. For instance, you may cumulate several targeting Accents to add additional effects. In conclusion, remember that \textbf{the GM always has the final say on how many Accents are required to create a particular effect}, and that they can contradict the rules as desired.


\subsection{Casting types}
\label{casting_types}

Spells are usually, but not always, opposed by the target's Will if they are hostile, which becomes the Difficulty of the Spell roll. The environment can of course apply a custom Difficulty (replacing Will) and in certain cases, for difficult non-hostile spells, a non-Will Difficulty can also be applied by the GM.

There are two main ways to cast Spells, with varying speed: \textbf{Snap Casting} and \textbf{Calm Casting}. As they are Abilities, casting a spell costs Focus\footnote{When Calm Casting, this Focus is refunded by the implicit short rest taking place.}. Material components may be required for powerful Spells, but this is more of an exception, as the plot demands.

\paragraph{Snap Casting}

Snap Casting is also called immediate casting. This is the kind of spellcasting used in combat, and as such uses the combat rules. This means your Aptitude will be your appropriate combat Accuracy: see section \ref{accuracies} for more details. 

That being said, it can be used out-of-combat when the Actor casting the spell is, generally speaking, in a hurry: when they have secondes to minutes to spend. Here, INT and Kosmics still play a role, since they serve as caps to your modifiers as discussed above.

\paragraph{Calm Casting}

Calm Casting, as the name implies, is the type of casting used when the Actor has a lot of time available, minutes to hours. For instance, Enchanting objects is a technically a kind of Calm Casting. This type uses INT and Kosmics directly, with the sum of the two becoming the Aptitude for the casting. Of course and as usual, additional modifiers may be added by the GM. Remember that the penalty for excessive Accents also applied here.

\section{Examples}

So far, we have discussed Spells on a very abstract level. Let us make it more concrete. Detailed examples of Spells, with their constituent Sigils and Accents, are presented in section \ref{examples}: feel free to refer to them when designing your own! Examples of crafted Objects and Traits (ie. custom Passive Abilities) are also presented to use as inspiration.

Generally, the key resource when designing and balancing spells are the considerations presented in sectoin \ref{balancing_spells}. The goal with the Accent system is to stimulate imagination instead of having a rigid list of prebuilt spells, so feel free to be creative!














