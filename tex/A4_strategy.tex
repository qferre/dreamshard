\chapter{Grandstrategy}

Those Grand Strategy, or "Grandstrategical" rules if you'll forgive this portmanteau, are \textbf{very optional}. Use them only if you wish to add a grand strategy component to your campaign, with geopolitics and military manoeuvers.

\label{strategy_rules}

The grand strategic map is divided into \textbf{provinces}. Depending on the scale of your campaign, I recommend that a province should cover an area with a diameter of roughly 80-100km in densely populated areas, a larger diameter in less populated ones, and a much larger diameter for the sea. Of course this can be scaled up or down. Each turn represents something like a few weeks, depending on the scale.

If you are using the same map as the ones used for your PC's travels, then instead I recommend that the provinces should be smaller and each turn represent only one week.

Players take their turns one after the other. In this instance, a "Player" designates a faction or more generally a side, not an individual Player Character: the PCs are one faction by themselves. The scenario designates which player goes first.

\begin{rpg-examplebox}
	As with the core rules, the Game Master may assign any modifier they wish to any roll, or change any rule, to account for specific circumstances.
\end{rpg-examplebox}


\section{Player turn}

A turn consists of two Phases: Economy and Action, taken in that order.

\begin{enumerate}
    \item \textbf{Economy Phase}: recruit Armies (and Agents), place Buildings, collect income, pay upkeep.
    \item \textbf{Action Phase}: Units movement and combat, and Agents can act.
    \item Pass to next player.
\end{enumerate}

Within a Phase, you can do the allowed actions in any order you wish: for example during the Action phase, you may use an Agent, then an Army, then another Agent, etc.


\section{Provinces}

Each controlled Province grants an income of 1G (Gold, representing resources in general) per turn at the beginning the Economy phase.

A Province is captured automatically when a Player's Units move inside it, unless troops from another Player are present: in this case, there is a battle (see later).


\subsection{Buildings}

\rpgart{t}{img/art/fort}

Each Province has 2 Building slots. During the Economy phase, you may pay 4G to build a Building in the first slot, or pay 6G to build in the second slot if the first slot is already occupied. The building is placed face down, and activated at the beginning of your next Economy phase.

Here are the standard Buildings:

\begin{itemize}
    \item \textbf{City}: Gives +2G per turn
    \item \textbf{Barracks}: allows Unit recruitment, and adds fortifications (+3 in a defensive battle)
\end{itemize}

A province with both a City and a Barracks will need to be sieged before being captured.

Relatedly, It should be noted that "Building" is actually a very general concept representing any kind of distinctive feature, as is explained in the Special Buildings section below.

\section{Armies}

An Army is a stack of Units, that can move and fight on the main map.

Armies have 2 move points per turn. Moving to an adjacent province usually costs 1 point\footnote{Army move speeds will need to be tweaked depending on how much time each turn represents, or the province sizes.}. Armies in friendly territory can move faster with an Agent action.


\subsection{Recruitment}

To recruit an Unit during the Recruitment phase, pick a Province with a Barracks. When recruited, the Unit is placed face down and will activate at the beginning of your next Economy phase. Each Barracks can recruit at most 2 Units per turn. Note that Units cannot be recruited in conquered territories until 3 turns after the conquest\footnote{Unless you are reconquering one of your core territories}.

You can either recruit generic Units at a cost of 1G each, or more costly Unique Units. Unique Units are created by the GM and will generally give a bonus if certain conditions are met. An example: "Heavy Cavalry, gives +1 in Late Phase unless the enemy has Pikes". Multiple Unique Units can be stacked in the same army, and their modifiers are cumulative.

Each regiment necessitates an upkeep of 1G per turn, during the Upkeep phase. 
Heroes require no upkeep. 

Fleets are required to cross naval provinces\footnote{In certain scenarios, Fleets may be abstracted by a maritime movement path between two provinces, but this is not the general case.}. Fleets are recruited with the same principle as land Units, except they cannot enter land provinces but may occupy port. The player simply declares that they wish to recruit a naval Unit instead of a Land unit. 


\paragraph{Battles}

When an Army encounters another Army belonging to an enemy player in the same province, they automatically initiate a Battle if possible. Note that each Army can initiate only one battle per turn, and as such cannot invade an occupied province if they already fought this turn, even if they have Movement left.

Each Battle consists of two rolls, with the usual "d20 + modifier" core rules; Those rolls are called respectively the Early and the Late roll\footnote{The Early phase represents manoeuvers, skirmishes and early contacts, while the Late phase represents the main engagement and pursuits}. Unique Units and Agents can apply a modifier to any of those rolls.

The roll is made by the attacker, and the modifier is equal to the difference in number of Units, plus the modifiers from Unique Units. Obviously, all of these are added positively for the attacker and negatively for the defender.

Whoever loses the first roll gets -2 for the second roll. After the Battle, each side loses 25\% of their Units per roll they lost and 5\% per roll they won rounded down. Each player chooses which of their units to remove). Here "Grazes" count as Failures, resulting in a symmetrical success table, for balance reasons.

An army that loses two rolls must retreat to their choice of an adjacent friendly province, and is destroyed if they cannot. In case of a draw, both Armies may stay in place and a new Battle is initiated next turn, or choose to retreat.

While any land battle, siege and naval battles can be resolved with the those rules, if you want to be a bit more involved there are rules for that in section \ref{operations}.


\subsection{Capturing a province} 

As mentioned before, Provinces are captured by moving an Army into them, but only if there are no defending Armies in it. This means that, for example, in case of a tie in a Battle you will not capture the province unless all defenders were destroyed.

When a province is conquered, the building in the second slot is destroyed.


\section{Setup}

When beginning the strategic scenario, the GM will decide on the initial placement of Armies, Buildings, Agents, etc. If the scenario does not specify it, each Player starts with no assets and 2G.

The Game lasts until the Victory Conditions specified by the GM are met.


\subsection{Special Buildings} 

As we discussed before, a "Building" is a very general concept, which is in fact closer to representing any "province feature" and may have its own rule.

For example, basic custom buildings may give different modifiers (mine, farm, temple, school, etc.). But, for example, a movable "building" could represent a guerilla effort in another Player's province, or the vassalage of local nobles. Terrain features may also be represented; as explained in the other strategic rules, terrain can affect movement, combat and economy.

\section{Agents}

\label{agents}

Agents are special Actors, and this also includes Heroes (Player Characters and other important \textit{dramatis personae}). They are represented as Pawns on the campaign map.

Agents are (N)PC and have a character sheet, but in practice, they will be described based on their role. For instance, a "+5 Spy" is an Actor with +3 DEX and 2 Discretion\footnote{Note that acting on a strategic scale has an inherently higher Difficulty}.

The cost to recruit Agents depend on their quality (modifier for their role), and may be given by the scenario. Indeed, in the latter cases, most Heroes can be Agents.

Agents move twice as fast as Armies. Armies can not intercept Agents in the general case, but it is possible to make a roll for it if there is a siege or a significant event.

\subsection{Agent Actions}

Each turn, a player may describe and the GM adjudicates 1 action per Agent during the Action phase.

Here are some examples of such actions: 

\begin{itemize}
    \item Lead an Army, granting combat bonuses.
    \item Spying to reveal intel.
    \item Assassinate another agent.
    \item Stimulate the local economy.
\end{itemize}

I think those should be present in most games, but feel free to add some beyond that. 

The GM decides what rolls each Action entails. They are adjudicated with the usual core roll rules: 1d20 + Aptitude, Difficulty and relevant modifiers. In terms of impact on the game board, it should be roughly equivalent to a battle.

For the PC's actions, you can use this principle as well, but I recommend using Snapshots as seen below. This is a Role-Playing Game after all!


\subsection{Autonomous Agents}

If the GM decides, NPC Agents can have an will of their own. Each turn, such an Autonomous Agent each turn, picks Action to do depending on their chosen Agenda. Agendas can be influenced by other Actors (including the PC), learning from the consequences of past actions, etc.



\section{Snapshots}

We must not forget that this is a Tabletop Role-Playing Game! As such, all events in the grand strategy campaign such as Battles or Agent Actions can have Snapshots. Those are key moments and scenes, played with the classical RPG rules, that are part of the scenario and are in fact a way to "zoom" from strategic to a RPG scene.

For example, a battle could be played using the operational combat rules mentioned in another section, and some key moments such as an infiltration or a last stand can be played in the RPG scale. In this case, the diegetic outcome of the Snapshot should influence the final result.

Indeed, for Agent Actions, if the PC are involves they can and should play them directly with RPG rules, as part of the scenario\footnote{It is always the case for the "main quest", at the GM's discretion.}.

The goal of these grand strategy rules is to be an accelerator of gameplay and a story generator: as such, the parts that are interesting to the story will be played as an RPG, and the tedious parts can be abstracted away with those rules.


\subsection{Deepening the rules}

Finally, if you zoom in even at the strategic map level, consider the Operations as presented in section \ref{operations}. When playing at the grand strategy level, most of these are abstracted away in the Movement Points, Buildings, etc. but can be worth keeping in mind, Reputations in particular.


\section{Special rules}

Building upon this framework, the GM can add any manner of Special Rules to represent certain aspects of situations.


\subsection{Faction bonus}

Some players (factions) may get some inherent advantages (equivalent of Traits) to represent cultural specificities. Examples: improved Unique Units, reduced Upkeep in certain conditions, faster constructions, better fortifications, can relocate buildings, etc.

\subsection{Events}

The scenario may have planned certain special Events, that can trigger if certain conditions are met and affect the game board and player status in any possible way: for instance, switch control of provinces, give modifiers, place new Pawns, modify diplomatic relations, natural events, etc. This is also a general category for any modification the GM may wish to make to the state of play.

Events are conceptually linked to Agent Actions (the result of such an Action is an Event). They can affect the board in any way (ie. camouflaging an Army can be an Event) and may represent not only military actoins, but also political and diplomatic maneouvering.

\subsection{Diplomacy}

There are no specific rules for diplomacy between human players. Diplomacy with other Factions will likely be mediated by Agent Actions. However, in general, Diplomacy will also likely be handled by the plot itself. 

Alternatively, it is very possible use something like the Reputations framework, presented in section \ref{reputations} and expanded for strategical concerns in section \ref{relations}: so that each faction can accumulate Reputation with another faction. Also consider that factions, like autonomous NPCs, can have Agendas which define explicitly their objectives, priorities, and what they will try to negotiate for. Reputations and Agendas combined quantify each faction's willingness to respectively accept and propose certain deals with each other.