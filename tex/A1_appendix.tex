\chapter{Gameplay}

In this section, you fill find any resource necessary for the gameplay. Notably, you will find the complete list of Abilities, as discussed in section \ref{abilities}.

\section{List of Abilities}
\label{abilities_list}

Abilities are given in the order in which they should be unlocked, from first to last; their position in the Ability Tree is called their Tier. 

When adjudicating the results, favor common sense as opposed to following the rules as written. For example, using Take The Hit cannot prevent an ally from dying of an illness, unless you are a skilled physician yourself. Furthermore, if a player is trying to use an Ability in a manner that is clearly ineffective, such as by using it against a target immune to its effects (and which the Player Character knows would be immune to it), be a good sport and warn them beforehand! The latter point applies to any gameplay action in general.

Recall that Conditions and their modalities are detailed in section \ref{conditions}. Also recall that Passive Abilities are always active and do not cost Focus.

\subsection{Red}

\subsubsection{Power}
\begin{enumerate}
    \item \textbf{Knock Down}: Make a regular Attack. If the Attack's result was a Graze or better, make a second roll against Fortitude to inflict Prone.
    \item \textbf{Clear Out}: Make a regular attack\footnote{You can use Clear Out in Overwatch to counter-charge automatically, meaning you charge and go right before the impact point of a another person's charge or movement.}. If you spent at least 1AP moving directly towards the target, during the same round, and managed to inflict damage, Clear Out inflicts 2+2\texttimes MIG additional damage. If the attack result was a Graze or better, then roll against Fortitude to knock the target back as if you had Wounded it. For the latter roll, Grazes are converted to Successes.
    \item \textbf{Fire Sigil}: Self-explanatory. Heat and combustion.
    \item \textbf{Fury}: Bestow upon yourself the Enraged Condition.
\end{enumerate}

\subsubsection{Defense}
\begin{enumerate}
    \item \textbf{Ever Vigilant}: \textit{Passive Ability}. If you spend at least one Action Point on Overwatch during your round, gain 1 more AP worth of Overwatch at the end of your round. The Overwatch AP thus granted cannot be spent to use an Ability.
    \item \textbf{Stalwart}: \textit{Passive Ability}. Critical Hits against you only have a \texttimes 1.5 damage multiplier\footnote{Criticals boosted by Opportunist have a \texttimes 2.5 multiplier.}. You are only Knockbacked by one tile instead of two.
    \item \textbf{Take the Hit}: Pick an Actor, 2 tiles away from you at most. You suffer the consequences of the next hostile Action inflicted against them in their place. Any damage taken is reduced by your Armor.
    \item \textbf{Telluric Sigil}: Telluric manipulation. This includes hardening to stone, transmutation, slickening with oil, etc. The difference between this and the Nature sigil is that Nature is more focused on ecosystems.
\end{enumerate}

\subsubsection{Zeal}
\begin{enumerate}
    \item \textbf{Into The Fray}: Pull an Actor (friend or foe) up to 3 tiles away into any tile adjacent to you\footnote{Abilities like Into The Fray work in 3D, but only if reasonable for the caster: for instance, a character whithout a grappling hook cannot pull someone from a square he would be unable to move to.}. Against a hostile Actor which is larger than you, or well-prepared, you will need to make a roll against Fortitude to succeed.
    \item \textbf{Action Surge}: \textit{This Ability does not cost an Action Point}. Pick any one Actor, including yourself. They immediately gain 1 Action Point\footnote{They can use this to act immediately out-of-turn, or conserve it for their next turn.}. This cannot be used on the same Actor more than once per turn.
    \item \textbf{Epiphany}: An ally gains a temporary +3 to a Skill or Accuracy where they already had at least +1, until the end of their turn or equivalent time. You cannot use this on the same Actor on the same or consecutive turns, and you cannot use it to boost a roll that is not an Attack roll more than once on the same Actor between short rests.
    \item \textbf{Indomitable}: \textit{Passive Ability}. Pick with the GM your Sworn Enemy\footnote{It can be an individual NPC in the campaign, or at most a small defined group.} based on your background. This Sworn Enemy has a permanent -3 penalty to all rolls against you.
\end{enumerate}

\subsection{Green}

\subsubsection{Guile}
\begin{enumerate}
    \item \textbf{Hobbling Strike}: Make a regular Attack. If the Attack's result was a Graze or better, then roll against Fortitude to inflict the Crippled condition. For the latter roll, Grazes are converted to Successes.
    \item \textbf{Opportunist}: \textit{Passive Ability}. Against Flanked or Incapacitated targets, you get an additional\footnote{Additional, meaning it is cumulative with the normal bonus of +3 when Flanking in melee, and with the +3 bonus of only using Light weapons, if either is applicable.} +4 Accuracy bonus. If you make an attack that benefits from Opportunist using only Light weapons, the damage multiplier on Critical Hits is \texttimes 3 instead of \texttimes 2.
    \item \textbf{Illusion Sigil}: Create illusory effects and influence the minds of your targets.
    \item \textbf{Dazzle}: Roll against Fortitude to inflict the Blinded condition against all adjacent enemies. This does not include a regular Attack ; however, on the turn you use this Ability you may also use a Gadget\footnote{Including, but not limited to, objects created via Crafting in section \ref{crafting}. In fact, "Gadgets" can designate any practical object if the GM so wishes.} one time without spending an additional AP.
\end{enumerate}

\subsubsection{Raiding}
\begin{enumerate}
    \item \textbf{Pin Down}: Make a regular Attack. If the Attack'q result was a Graze or better, make a second roll against Fortitude. If the latter roll is a success, the target loses 1 AP next turn (this effect cannot stack with itself). For the latter roll, Grazes are converted to Successes.
    \item \textbf{Mayhem}: \textit{This Ability does not cost an Action Point, but can only be used once per turn.} Use a Gadget. For this Action, you gain the same bonus as if you had been Concentrating (usually a +2).
    \item \textbf{Atrophy Sigil}: Inflict negative Conditions on opponents, resulting in various penalties. Conditions applied by Atrophy will generally be broader than conditions applied by other Sigils\footnote{For instance, Atrophy may lower all Accuracies, whole other sigils will only lower a specific one. Atrophy may also lower all Defenses, lower Armor, etc. In counterpart, other Sigils are more polyvalent. I am considering tying the Condition applied to certain Accents.}.
    \item \textbf{Called Shot}: \textit{Passive Ability}. Whenever you make an Attack, you may aim for a specific body part \footnote{Recall that Criticals when aiming for a specific body part inflict a small Condition.} without penalty (or +2 if you would not have take a penalty without Called Shot).
\end{enumerate}

\subsubsection{Insight}
\begin{enumerate}
    \item \textbf{Emotion Sigil}: Manipulation thereof, for friends or foes. To both inspire and hinder.
    \item \textbf{Sixth Sense}: Pick any one Actor within 4 squares, including yourself. They may reroll one roll of their choosing during their next turn, and keep the most favorable result. You cannot use this on the same Actor on the same or consecutive turns, and you cannot use it to boost a roll that is not an Attack roll more than once on the same Actor between short rests.
    \item \textbf{Coordinate}: You may have a follower, be it an animal familiar, a henchman, or anyhting that makes sense for your character. Create it with the GM. It must be six levels under you, and will have a more limited selection of Abilities and Skills\footnote{For players of level $\leq$ 6, de-level the follower compared to a regular PC by removing as many Characteristic Points, Skill points, etc. as would have been gained. The GM has the last say on what to remove, and will generally prioritize capping its highest Characteristic at a lower value.}. You and your follower can switch turns in the Initiative queue at-will, but this cannot cause either of you to act twice in a turn.
    \item \textbf{Nature Sigil}: Influence natural elements and ecosystems, including the biosphere and the weather.
\end{enumerate}


\subsection{Blue}

\subsubsection{Elementalism}
\begin{enumerate}
    \item \textbf{Frost Sigil}: Self-explanatory. Ice and lower temperatures.
    \item \textbf{Lightning Sigil}: Self-explanatory. The Electricity Fairy at your fingertips.
    \item \textbf{Sigil Merging}: Combine any two Sigils that you know. They can share the first Power Accent, but other Accents must be added and paid for individually\footnote{Merged Sigils can result in new effects. However, for balance reasons, they can detonate Combinations but they cannot prime them. For example, a merged water and lighting spell is going to electrocute someone who was already wet thanks to its lightning component, but is not an automatic Critical on a target which was not wet, as it was not 'primed' (ie. not chilly, wet, etc.).}.
    \item \textbf{Wildcard}: When you learn this Ability, two things happen: (1) you may pick one Sigil from anywhere in any tree and learn it. Also, (2) you immediately gain 2 Skill points to spend.
\end{enumerate}

\subsubsection{Blessings}
\begin{enumerate}
    \item \textbf{Vigor Sigil}: Grant bonuses to Defenses and Armor. May grant bonus to specific\footnote{By specific, here is the kind of broadness I mean: ie. 'all attack rolls' or 'all social rolls'} rolls with sufficient Power.
    \item \textbf{Life Sigil}: Perform Healing to restore HP and cure Wounds.
    \item \textbf{Second Wind}: Pick any one Actor, including yourself. They regain 4 Focus. Focus over the maximum is lost, but only at the end of their next round. This Ability cannot be used more than once on the same Actor between short rests.
    \item \textbf{Dispel Sigil}: Roll against the caster's Will to cancel a Spell. Can cancel an ongoing Spell, or be used as an Overwatch action to counter a Spell being cast.
\end{enumerate}

\subsubsection{Spiritism}
\begin{enumerate}
    \item \textbf{Enchanter}: \textit{Passive Ability}. When Enchanting an object with a Sigil (see Enchanting at section \ref{enchanting}), gain an additional two free Accents. Only one of the two Enchanter accents can be a Power Accent.
    \item \textbf{Force Sigil}: Telekinesis on objects or people, yourself included. Pushing or pulling, anything is fair game. Heavily reliant on Accents.
    \item \textbf{Divination Sigil}: Prescience and advanced calculation. Also used to communicate with magical entities.
    \item \textbf{Summoning Sigil}: Summon manifestations and echos to influence the physical world. Also covers necromancy and construct (golem, robot) creation.
\end{enumerate}

When it comes to Spells, additional clarifications about the effects and limitatoins of Magic is given in the Lore Appendix, in section \ref{magic_effects}.


\section{Spell Accents}
\label{spell_accents}

As discussed before in section \ref{spells}, a Spell is made by adding Accents to a core Sigil. This will determine the modalities of the Spell. 

In this section, Accents are sorted by type for easier parsing. \textbf{Most Spells will require at least one Power Accent and at least one Targeting accent.} Effect modulation accents are not mandatory. Otherwise, this distinction is purely for convenience and has no other intrinsic gameplay effect. Depending on the GM, some Accents may need to be learned before they are available, but this is not the general case.

Generally, \textbf{see the section about Balancing Spells} at \ref{balancing_spells} for more information about balancing and designing all of this.

\subsection{Main}

\begin{itemize}
    \item \textbf{Power}: This Accent will be always applied $n \in \mathds{N}$ times. For damage (or healing), each Power Accent adds 2 damage. For modifiers applied (ie. when inflicting Conditions), each Power accent increases the magnitude by +1. (resp. for negatives). Power Accents will also increase the magnitude of the spell for absolute effects (ie. effects that are not directly numerical). See below for details.
    \item \textbf{Penalty}: The GM can impose Penalty Accents for effects that are not already covered by other Accents. But this can also be used to add flexibility: perhaps you may, for example, use an additional Accent to boost duration of a spell by 1. \textit{This is a tool for the GM to balance Spells outside of the Accents described here.}
\end{itemize}

\paragraph{Free Accents}

\textbf{The first Power Accent is free} and included in the Sigil, and does not count against the limit of Accents you can apply before suffering penalties. Furthermore, it does \textbf{3 damage instead of 2}.

This means that, when adding Power Accents, the damage progression is 5-7-9-etc.\footnote{Remember to add MIG regardless of power accents.} and the magnitude progression for Conditions is 2-3-4-etc.

The Targeting Accent "On Touch" is also free, see below.


\subsection{Targeting}

\begin{itemize}
    \item \textbf{On Touch}: Affect an Actor by touching them. \textit{This is a free Accent}, but in counterpart it cannot be used to implement effects that are covered by another Accent. For example you cannot use this to affect yourself, or affect an object.
    \item \textbf{On Self}: Apply the effects of the spell to the caster.
    \item \textbf{Impact}: Affect a distant target. Each Impact Accent adds 5 tiles to the Spell's range, starting at 0 and maxing at 15. 
    \item \textbf{Area of Effect}: The Spell will affect (or try to) all targets within a certain radius of the point of impact. Friendly fire is possible! The blast radius is one tile per such Accent, maxing at 4.
    \item \textbf{Spray}: Affect an adjacent tile from the point of impact and the 3 tiles directly behind it. Friendly fire is possible!
    \item \textbf{Additional target}: Aim preciselly for one additional target per such Accent. Unlike AoE Accents, this prevents friendly fire.
    \item \textbf{Affect Object}: Apply an effect on an object. For example, add an effect to a weapon. This is akin to Inscribing, but done live.
    \item \textbf{Homing}: Allows you to use Ranged Accuracy for a melee Spell, or vice versa.
\end{itemize}

Targeting Accents can be combined. For example, Fire + Impact + Area of Effect results in a fireball. Spray + On Touch is a classical spray, but Spray + Impact is like a bursting shell. It is also possible to mix and match : Fire + Area of Effect + Impact + Additional Target will make a fireball, but will also have one supplementary target outside the blast radius.

\subsection{Effect modulation}

\begin{itemize}
    \item \textbf{Duration}: The spell has effect over time. If you make a spell without a Duration accent, any Condition will only last for 1 turn. If you add a Duration accent, then it will use the general rule: it  lasts for INT-of-the-caster minus RES-of-the-target. Recall (see in Balancing) that damage and healing over time have half the magnitude of bursts.
    \item \textbf{Channeling}: Continously channeling a Spell. Conditions' countdowns don't start until the end of the channeling, and damage or healing is repeated at the beginning of each round until the end of the channeling. As such, the casting becomes an Ongoing Action as explained in the Combat rules\footnote{And can be interrupted by failing the corresponding Fortitude test. Singing will not reserve AP, and as such its effects are more limited, but focusing a scorching ray with your hands will, and as such be more powerful.}.
    \item \textbf{Reaction}: Do something when something else happens. This is used to design spells that work like traps or Overwatch, meaning they have a trigger.
    \item \textbf{Stealth}: Discreet spellcasting, without any external signs a Spell was being cast until its effects manifest.
    \item \textbf{Tradeoff}: Successes are converted into Critical Successes, but Grazes and Misses are converted into Fumbles.    
\end{itemize}

Tangentially, if the GM wishes it, a Fumble when using Magic may result in a different effect altogether than what was intended (called a "Magic Surge"), instead of simply friendly fire or failure. Some Spells may also have a list of effects to draw from, at random or not.