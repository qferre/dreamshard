
\chapter{Interactions}

In the general case, any interaction involving at least one Actor such as convincing someone, disarming a trap, etc. will use the core rule as presented in section \ref{dice_rolls}: namely, make a dice roll as an Aptitude vs Difficulty test. This applies whether an Actor intereacts with other Actors, objects, or the world itself. There is more nuance to this when it comes to combat, and other special cases have their own special rules, as we explained previously. But when in doubt this basic principle should guide you. The Difficulty can be custom (eg. for traps), or depend on the Actors' attributes.

In this section, we will cover more specific edge cases that were not discussed in the previous parts.

\paragraph{Passive tests}

Passive tests are used when assessing an Actor's general awareness of the situation. For example, do the players find this hidden object when they not are actively looking for something, or do not know there is something to be found? To represent this, simply decide on an Aptitude vs Difficulty roll, and assume the result of the roll was 10, as explained in the core rules.

A perception test will generally use \textbf{WIT}. Adding a Skill (usually Discretion) may be justified in certain cases. This can also be relevant for knowledge checks.

\section{Stealth}

To move discreetly (including in combat, but this is more difficult), the Aptitude is \textbf{DEX + Discretion} skill, and the Difficulty will be the observer's WIT. If the observer is being vigilant, the Difficulty is instead WIT + Discretion.

Stealth rolls are amongst those most heavily influenced by circumstancial modifiers. For instance, being careful to stick to the shadows can give +3, while being too close or potentially in the field of vision of a sentinel can give -3 or more.

Relatedly, looking for clues in general can also involve WIT (+ Discretion or other relevant Skill), but INT can be used instead in an analytical investigation (+ any relevant Skill as usual).


\section{Protracted Contests}
\label{protracted}

The general way to resolve an interaction involves a single dice roll. Recall that direct, frontal opposition uses the core rules for Opposing rolls (\ref{dice_rolls}). But sometimes, for contests that are more impactful or simply carry on longer, we may wish for a more... involved rule. A simple way is to ask for several rolls (say, three), and counting the number of successes among those.

There is also the Contest method, presented here. I recommend it be used sparingly, for set piece moments, such as convincing a grand assembly or a final boss. Here is the procedure:

\begin{enumerate}
    \item Set a number of Steps. Each Step will correspond to a roll.
    \item Set a total Challenge for the Contest.
    \item Decide on the Value system: classically, the value of each Step is the total of the roll for this step (result of the roll plus Aptitude and modifiers and minus Difficulty, which is the point 5 in section \ref{dice_rolls}) but you may have preset Values depending on the success quality instead\footnote{Useful if you want to give the players explicit choices between different approaches at each Step, such as a safe one versus a high-risk-high-reward one.}.
    \item Perform a Step, until the total number of Steps is reached: ask the players what they do, decide what the corresponding Aptitude, Difficulty and other modifiers are (they may be different for each Step), and make the roll. Accrue Value depending on the outcome of the roll and the Value system chosen.
    \item If the \textbf{total Value accrued over the Steps} exceeds the Challenge, the Contest is won, otherwise it it lost.
\end{enumerate}


\section{Reputations}

\label{reputations}

Actors, particulary Player Characters, will have Reputations with other parties in the world. Here, a "party" is to be understood in the broadest sense: it can cover a single person, a group of friends, a secret society, an entire poltical faction, etc.

Reputation comes in two varieties: \textbf{positive and negative}. I recommend that those be tracked separately, each from a scale of 0 to 3. Each can increase and decrease separately\footnote{For obvious reasons, we expect them to be somewhat anticorrelated.}. Unsurpringly, these Reputations will depend on the Character's past deeds and the view this party has of the Character. The GM decides at which points all of this adds up to crossing one (or more) Reputation tiers.

As a consequence, a Character with a Reputation of 0-0 one with 3-3 will not be seen in the same way: the former is an unknown quantity, while the latter is known as a very unreliable element.

For the "small" people, and people beholden to you, positive Reputation with them is called \textit{Favor}, or \textit{Loyalty} if applicable, while negative Reputation is called \textit{Fear}. However, for powerful people and groups, positive Reputation with them is \textit{Favor} while negative reputation is \textit{Wrath}.

Reputation can effect certain perks. This is mediated by the group and the GM. However, be careful: if the GM decides, relying on your reputation and calling on favors too often can reduce a Favor tier!



\section{Operations}
\label{operations}

This section covers elements to consider when the Player Characters will be travelling, and generally operating in the world on a larger scale. These rules will let you send in the operators to operate operationally, but they should be considered very \textbf{optional}.

For truly "grand strategy" rules, meaning military, political and economic manoeuvering, another simple set of optional rules is presented in section \ref{strategy_rules}. But large-scale maps will not necessarily need to use those grand strategy rules! In most campaigns, the player characters will just be travelling and engaging only in a limited manner with the geostrategic landscape. As such, you can simply use a map of the region\footnote{I would still recommend dividing it into provinces representing each a few days of travel time, as presented in the strategic rules section, instead of squares; this is more organic.} and use the rules presented in this Operations section.

Parenthetically, \textit{there can be cross-pollining between both sets of rules}: the Operations rules can also be applied to the Grand Strategy rules in an ad-hoc fashion, as needed, mostly for appropriate travel times and supplies rules presented below. And vice-versa, a more limited subset of Grand Strategy rules can give a large scale map more depth.


\subsection{Travel times}

On short bursts (sprinting), a horse is roughly twice as fast as a human. That being said, long distances are another matter! You can ride a horse for roughly 50 kilometers per day on average, but the horse will not be able to do so for more than a handful of days without resting. However, humans can walk 20-30 km per day without much trouble, but crucially they can keep this up over weeks. Both human and horses can double these daily rates on emergencies, but this will be exhausting.

At sea, a typical (large) ship will sail at 10-20 km/h, with a maximum speed of 20-30 km/h, and cover 200 km per day, assuming the winds are not unfavorable. A small ship can be faster, with a top speed of 30-40 km/h.

For simplicity, consider in most cases that difficult terrain such as a dense forest, mountains, or crossing a river, doubles the travel times mentioned above. You can use a lower or higher modifier depending on the specifics of the terrain: for instance, an absolute quagmire of a jungle swamp with no trails will likely take \texttimes 4 time instead of simply double, while gentle hills would only apply a \texttimes 1.5 penalty, and good roads will instead quicken your travel by applying a \texttimes 0.75 modifier to travel time.

\subsection{Warbands}

Marauding warbands, or combat groups more generally, may be represented by Armies like in section \ref{strategy_rules}, with a chance to avoid engagements with other Armies due to their small size. Otherwise, they can just be represented by preset pawns that move along on the map. Warbands will generally travel somewhat slower than lonely adventurers or small groups, but benefit more from roads.

Supply is a critical matter, be it for a warband of even a handful of individuals. Resupplying can be done at a market or via looting and foraging. Supplies and supply consumption is also discussed in section \ref{equipment}; it will usually cover food, but also any tool, weapon, etc. To simplify, Actors should \textbf{gain Wounds if they are out of supply for more than 2 days}. Generally, one may carry enough supplies for a few days on their person, but carrying enough for a prolonged expedition will absolutely require wagons and pack animals. Optionally, keep in mind that supply and demand will cause prices to vary, and cause caravans to form to carry what is needed where it is needed\footnote{Carvans function like a Warband, and Supply availability will be impacted if they are lost.}.
	

\subsection{Relations}

\label{relations}

This builds upon the Reputations presented above in section \ref{reputations}. As mentioned there, Relations are measured in Favor and Wrath usually, but Favor can represent Loyalty instead, for subordinates.

When talking to other Actors, having a lot of Favor, low Wrath, or being seen as honorable will all facilitate negotiations\footnote{Or make people more willing to surrender... Just sayin'.}. Conversely, Actors with low Loyalty\footnote{Or which are more generally dissatified with you.} may betray you.

Diplomacy and level-headed negotiations will mostly rely on INT, but charm and RES have their uses in negotiating alliances, particulary dynastic ones\footnote{Everybody likes a good love story, don't they? Especially if there are huge... tracts of land involved.}. Leadership and military organization will benefit from RES and INT respectively; this helps raise and lead troops in combat. Everything in this paragraph can also benefit from an appropriate (Special) Skill.

Agents can hold Titles, which represent responsabilities but also potentially rule over parts of the world and other characters. Title attribution (and potential inheritence) will depend on the society or potentate granting them, and said supreme authority will have demands in return. This is especially relevant for the Grand Strategy rules of section \ref{strategy_rules}.



\subsubsection{Overworld}


\rpgart{t}{img/art/naval}

As discussed in section \ref{hazards}, the weather and the seasons can facilitate or hinder movement, resource availability, or certain rolls in general\footnote{A few things to keep in mind: storms (non-magic ones, at least) don't last, because air masses are mobile, especially under the ocean's influence. Wind cells are a thing, and mountain chains will stop them. Mountains chain placment obeys the tectonic plates. Finally, an increase in altitude and latitude can quickly chill the climate.}. 

New recruits can be levied in willing townships or fortified places. Mercenaries are also an option. In those times, 1 hectare (10 000 square meters) worth of agriculture can support a family, and 10 workers can roughly support one armed fighter, but in practice less than 1\% of a population will be under arms at any time since money, troop quality and supplies are a far more important limiting factor. Remember that population fluctuates a lot.

Any information you will have about the world will always be imperfect, and only valid at time of collection. Hence, do not undestimate the importance of sending scouts ahead. Relatedly, at ground level, a human can only see about 5 kilometers (due to the Earth's curvature) if there are no obstacles and the weather is clear. Tracks on the ground can remain for days, again depending on the weather. Magic can help alleviate these limitations. 

Sending messages, orders, etc. by rider or ship will takes time: recall the maximum speeds we discussed above for humans, ships and horses. Magical communication, while possible, is rare and costly and will be reserved for critical communications. That being said, even without magic, messenger pigeons can be exceptionally effective, able to cover distances of up to 1800 kilometers, and doing so in three days!

Constructing a building will take a long time, which will of course depend on how many workers are available, and on the size of the construction. A legion can build a camp in a day, but a cathedral will take decades to centuries. Most civilian structures would take a few weeks to build, while castles take years. Relatedly, stone ruins can remain in the landscape for a very long time, but they will usually be quickly scavenged for materials. Said building materials will have different resistances\footnote{And potentially opacities. On that note, If you want to go nuts with a building system by filling each battle map tile with a different material voxel, be my guest, but be careful. Hell, if you want to use these voxels to build logic pipes with information flow and logic gates modifying the flow, you can. Can you tell what this is an Easter Egg for?}.



\subsection{Fighting as an army}

If you want to fight battles between larger armies, I recommend using a "zoomed-out" tactical map, hereafter called an operational map. Here, each tile represents enough space for a whole battalion of people to stand on. Such battalions are treated on this operational map as a Swarm would be (see section \ref{actor_size}) on the tactical map. I would recommend grouping no more than 100 human-sized creatures in each company\footnote{In their case, of course, the total HP does not necessarily account for the alive soldiers, merely for the battle-ready ones.}.

Otherwise, they should are treated as a "superlarge" specimen of their creature type, with the same general abilities. The beauty of this approach is that, on this operational scale, you can still mostly use the tactical rules just-as-written, notably for Knockback and Flanking.

Here, important single Actors (including but not limited to PC) are called Agents. In such an operational scale, Agents do not occupy their own square but should embed in units, providing bonuses as described later. If it is necessary to use them, use tactical RPG battles instead, representing setpiece moments of the battle.


\paragraph{Hirelings}

As mentioned, it is possible to recruit auxiliaries for the Player Characters. Those can have the Character Sheet of a level 1 character or weaker. Importantly, they can become strategic Agents (see section \ref{agents}), even if you do not use the rest of the Grand Strategy rules.

These hirelings can range from supply carriers to spies, but their availability (and price) is far from guaranteed. This is entirely up to the GM and the story he wants to tell.

For balance reasons, if the players wish to have a very reliable follower, or one that will follow them in combat, this necessitates the Coordinate Ability (see section \ref{abilities_list}). Note that this framework covers hirelings but also animal companions, magical creatures, etc. Anything that is appropriate for the Player Character(s) doing the hiring.

\subsection{Sieges}

The average fortress will hold against a siege for 2 to 4 weeks. Recall that attrition (mostly disease) takes its toll on both sides. Building siege weapons takes a few days for half a dozen people, and artillery will need days to batter down walls, although cannons are more effective. Do remember that there is usually looting after a siege.


\paragraph{On the battlefield}

When trying to scale a wall, climbing ropes or ladders is equivalent to moving through a difficult terrain. Siege towers and rams work a bit like the ships below: they are large engines which have a crew. Castle gates have 40+ HP and Heavy Armor.

Actors on the wall gain +50\% range thanks to their elevated position. Walls also provide a helfy Deflection bonus (between +4 and +6) when shooting at troops on the walls, or against defenders when storming the walls, but not once the attackers are on the walls themselves obviously.


\subsection{Naval battles}

In this section, I'd like to present some quick draft rules to manage naval engagements. Ships are large Actors, which have a crew that moves along with them. The crew can take a variety of Actions, presented in this section, but it should be noted that this requires proper training and experience. If the PCs get involved in those manoeuvers, this will require certain Special Skills, and the GM may call for skill checks.

\paragraph{Movement}

The key factor determining a Ship's potential speed will be the wind direction. For simplicity, it will be either North, South, West or East. A ship can move 100\% of its movement capacity in the wind, 66\% in an adjacent direction, and 20\% against\footnote{Different riggings, such as Latin sails, may have different speeds.}. Launching or stopping a Ship takes a few turns, during which the ship moves at half speed. Much like large monsters, ships cannot turn on a dime and will have a turning radius (larger with speed).

\paragraph{Attacking}

Attacks are handled by d20 rolls like as presented in the core rules. It is possible to make a Called Shot on a specific part of a target: its crew, its hull, or its masts. Otherwise, the Attack defaults to aiming for the hull.

In terms of Critical Effects this can result in, the two main possibilities are starting a fire (the ship loses HP per turn until it's put out) and damaging the propulsion (the ship is immobile for several turns). 

\paragraph{Contact}

When two ships are close enough to throw grapples at each other, it is possible for one ship to move against another and initiate a Boarding. At which point, it becomes a tactical fight with the tactical combat rules if the PCs are involved. In order to jump between the ships, a [DEX + Athletics] roll is required. Relatedly, a ship can Ram another by colliding it during a movement, which will deal damage and automatically trigger a Boarding. 

\paragraph{Ship variants}

In terms of scale, small ships such a caravel, a sloop or a brig are typically 25-50 meters long. Larger ships, such as galleons and Frigates, are larger than that. Beyond their class, ships can take on and be outfitted for many roles: assault, artillery, fireship, troop transport, etc.








